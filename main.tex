%%%%%%%%%%%%%%%%%%%%%%%%%%%%%%%%%%%%%%%%%%%%%%%%%%%%%%%%%%%%%%%%%%%%%%%%%%%%%%%%%%%%%%%%%
% Dieses LaTeX-Vorlage dient für die Verwendung der Arbeiten (ATL, Diplom) gemäss
% den hf-ict Richtlinien (Version 3.0) zum Zeitpunkt der Erstellung.
%
% Diese Vorlage basiert auf früheren Arbeiten von:
% Steve Gunn (http://users.ecs.soton.ac.uk/srg/softwaretools/document/templates/)
% Sunil Patel (http://www.sunilpatel.co.uk/thesis-template/)
% Matteo Delucchi (https://github.com/matteodelucchi/ZHAW_thesis-template)
%
% hf-ict spezifische Änderungen wurden vorgenommen von:
% Stefan Baumgartner
% 
% Vorlage Lizenz:
% CC BY-NC-SA 3.0 (http://creativecommons.org/licenses/by-nc-sa/3.0/)
%%%%%%%%%%%%%%%%%%%%%%%%%%%%%%%%%%%%%%%%%%%%%%%%%%%%%%%%%%%%%%%%%%%%%%%%%%%%%%%%%%%%%%%%%

%----------------------------------------------------------------------------------------
% DOKUMENT SPEZIFIKATIONEN
%----------------------------------------------------------------------------------------
\documentclass[
    12pt,                       % Standard Schriftgrösse
    oneside,                    % Standard: Einseitiger Einband.
    ngerman,                    % Sprache. Verwende ngerman für deutsch (Neue Rechtschreibung). Verwende english für Englisch.
    %singlespacing,             % Abstandsoption: singlespacing, onehalfspacing or doublespacing (einfacher Abstand, halber Abstand oder doppelter Abstand)
    %nolistspacing,             % Abstand in Listen auf einfach setzen
    %draft,                     % Aktivieren des Entwurf-Modus: keine Bilder, keine Links und überfüllte "H-Boxen" werden angezeigt
    liststotoc,                 % Liste der Abbildungen/Tabellen/etc. in das Inhaltsverzeichnis einfügen
    %toctotoc,                  % Das Hauptinhaltsverzeichnis in das Inhaltsverzeichnis aufnehmen
    parskip,                    % Vertikalen Abstand zwischen Absätzen einfügen
    %nohyperref,                % Links im gesamten Dokument deaktivieren
    headsepline,                % Eine horizontale Linie unter der Überschrift anzeigen
    %chapterinoneline,          % Kapitelüberschrift und Kapitelnummer in eine Zeile stellen
    consistentlayout,           % Gleiches Layout für spezielle Kapitel
                                % Erklärung, Zusammenfassung und Danksagung
]{Diplomarbeit}

% Formatierungsinformationen

\usepackage{titlesec}
\usepackage[ngerman]{babel}
\usepackage{arydshln}
\usepackage{multicol}
\usepackage{pdflscape}
\titleformat*{\section}{\Huge\bfseries}



% Unkommentieren Sie die folgenden Zeilen, um nur eine Teilmenge von Kapiteln einzuschließen
% Dies ist nützlich für lange Dokumente, da der Satz etwas Zeit in Anspruch nimmt
%\includeonly{
%    Front/titlepage,
%    Front/imprint,
%    Front/abstract,
%    %Front/declaration,
%    %Front/acknowledgements,
%    %Front/symbols,
%    Chapters/Chapter1,
%    %Chapters/Chapter2
%}


%----------------------------------------------------------------------------------------
% PRÄAMBEL: PAKETE UND KONFIGURATIONEN
%----------------------------------------------------------------------------------------
% !TEX root = main.tex

%----------------------------
%   SCHRIFTARTEN UND 
%----------------------------

% Support for special characters
%\usepackage[utf8]{inputenc}    % Specify input encoding
%\usepackage[T1]{fontenc}       % Specify font encoding

% Set main fonts
% Fonts catalogue: https://tug.org/FontCatalogue/
% Wichtig: für Arial ist LuaLaTex notwendig!

\usepackage{fontspec}
\setmainfont{Arial}

%Font für Listing
\newfontfamily{\lstsansserif}[Scale=1.0]{Arial}

%----------------------------
%   Environments
%----------------------------

\usepackage{caption}           % Customized caption
\usepackage{subcaption}        % Subfigure captions
\usepackage{makecell}          % Per-cell formatting in tables (\makecell)
\usepackage{pdfpages}          % Required to include PDF files/graphics (\includepdf)

\usepackage{todonotes}         % Introduces the command \todo
\setlength{\marginparwidth}{2.5cm} % Adjust this if the todo notes are out of margins

% Create boxes as follows:
% \begin{colorbox}{red}{2}
\usepackage{tcolorbox}
\newtcolorbox{textbox}[2]{
    arc=3pt,
    boxrule=#2pt,
    colback=#1!25!white,
    width=\textwidth,
    halign=left,
    valign=center,
    colframe=#1!75!black
}

%----------------------------
%   Colors
%----------------------------

% Set up colors
\usepackage{xcolor}
\usepackage{colortbl}
% ZHAW Blue: Pantone 2945 U / R0 G100 B166
\definecolor{zhawblue}{rgb}{0.00, 0.39, 0.65}
% Colors related to code listings
\definecolor{codegreen}{rgb}{0,0.6,0}
\definecolor{codegray}{rgb}{0.5,0.5,0.5}
\definecolor{codepurple}{rgb}{0.58,0,0.82}
\definecolor{codebackground}{rgb}{0.93,0.94,0.95}

%----------------------------
%   Code listings
%----------------------------

% Setup code listings
\usepackage{listings}
\lstdefinestyle{mystyle}{
    backgroundcolor=\color{codebackground},   
    commentstyle=\color{codegreen},
    keywordstyle=\color{magenta},
    numberstyle=\tiny\color{codegray},
    stringstyle=\color{codepurple},
    basicstyle=\lstsansserif,
    breakatwhitespace=false,
    breaklines=true,
    captionpos=b,
    keepspaces=true,
    numbers=left,
    numbersep=5pt,
    showspaces=false,
    showstringspaces=false,
    showtabs=false,
    tabsize=4
}
\lstset{style=mystyle}

% minted is an alternative code listing package. (See chapter 2)
% For it to run successfully, ensure the following:
% - the Python package Pygments. Install with the following command:
%       python -m pip install Pygments
% - pdflatex (or xelatex) is executed with the flag --shell-escape
%   If you are using a TEX editor, you can modify the typesetting 
%   command somewhere in the settings.
%\usepackage[outputdir=build]{minted}
%\usemintedstyle{xcode}
% For fancier coloring schemes, see here:
% https://tex.stackexchange.com/questions/585582
% One could also create an own style in Pygments
% https://pygments.org/docs/styles/#creating-own-styles

%----------------------------
%   References
%----------------------------

% Set up references
\usepackage[
    backend=biber,             % Use biber backend (an external tool)
    sorting=nyt,              % Sortierung nach Name, Jahr, Titel
    style=apa                  % Choose here your preferred citation style
]{biblatex}
\DeclareLanguageMapping{ngerman}{ngerman-apa} % Sprache in Deutsch
\addbibresource{quellen.bib}   % The filename of the bibliography

\usepackage[autostyle=true]{csquotes} 
                               % Required to generate language-dependent quotes 
                               % in the bibliography
\DefineBibliographyStrings{ngerman}{%
  retrieved = {Abgerufen},
  from = {von}}

\DeclareFieldFormat{url}{\url{#1}}
\DeclareFieldFormat{abstracturl}{\url{#1}}
\DeclareFieldFormat{urldate}{#1}

\renewbibmacro*{url+urldate}{%
  \ifthenelse{\(\iffieldundef{url}\AND\iffieldundef{abstracturl}\AND\iffieldundef{abstractloc}\)\OR\NOT\iffieldundef{doi}}
  {}
  {\ifthenelse{\iffieldundef{abstracturl}\AND\iffieldundef{abstractloc}}
    {}
    {\printtext{\bibcpstring{abstract}}\addspace}%
    \printtext{\bibstring{retrieved}}%
    \setunit{\addspace}%
    \iffieldundef{urlyear}
    {}
    {\printurldate%
      \setunit*{\addcomma\space}}% hier Komma einfügen
    \printtext{\bibstring{from}}%
    \setunit*{\addspace}%
    \printfield{urldescription}%
    \setunit*{\addcolon\addspace}%
    \iffieldundef{url}{}{\printfield{url}\renewcommand*{\finentrypunct}{\relax}}%
    \iffieldundef{abstractloc}{}{\printfield{abstractloc}\renewcommand*{\finentrypunct}{\relax}}
    \iffieldundef{abstracturl}{}{\printfield{abstracturl}\renewcommand*{\finentrypunct}{\relax}}}}

%----------------------------------------------------------------------------------------
%   MARGIN SETTINGS
%----------------------------------------------------------------------------------------

\geometry{
    % Muss gegebenfalls angepasst werden. Die Masse wurden mit einem Ausdruck überprüft.
    paper=a4paper,      % Change to letterpaper for US letter
    inner=2.5cm,        % Rand innen -> Gemäss Definition 3.0cm
    outer=1.5cm,        % Rand aussen -> Gemäss Definition 1.5cm
    top=2.5cm,          % Rand oben -> Gemäss Definition 3.0cm
    bottom=2.0cm,       % Rand untern -> Gemäss Definition 2.5cm
    bindingoffset=0.5cm, % Binding offset
    %showframe,         % Show the type block of the page
}
\setlength{\parskip}{1em}
\usepackage{enumitem}          % Layout control for list environments (e.g, itemize)
%\setlist{noitemsep}           % Suppress extra spaces between items
%\setlist{nosep}               % Suppress spaces before/after list environments

%----------------------------------------------------------------------------------------
%   OTHER SETTINGS
%----------------------------------------------------------------------------------------

\newcommand{\keyword}[1]{\textbf{#1}}
\newcommand{\tabhead}[1]{\textbf{#1}}
\newcommand{\code}[1]{\texttt{#1}}
\newcommand{\file}[1]{\texttt{\bfseries#1}}
\newcommand{\option}[1]{\texttt{\itshape#1}}

%----------------------------------------------------------------------------------------
%   MS WORD FARBEN (ÄHNLICH)
%----------------------------------------------------------------------------------------
\usepackage{colortbl}

%Grün, Akzent 6
\definecolor{MSOGrünAk6}{RGB}{112,173,71}
\definecolor{MSGrünAk6H80}{RGB}{226,239,217}
\definecolor{MSGrünAk6H60}{RGB}{197,224,179}
\definecolor{MSGrünAk6H40}{RGB}{168,208,141}
\definecolor{MSGrünAk6D25}{RGB}{83,129,53}
\definecolor{MSGrünAk6D50}{RGB}{56,86,35}

%Blau, Akzent 5
\definecolor{MSBlauAk5}{RGB}{91,155,213}
\definecolor{MSBlauAk5H80}{RGB}{222,234,246}
\definecolor{MSBlauAk5H60}{RGB}{189,214,238}
\definecolor{MSBlauAk5H40}{RGB}{156,194,229}
\definecolor{MSBlauAk5D25}{RGB}{46,116,181}
\definecolor{MSBlauAk5D50}{RGB}{31,78,121}

%Gold, Akzent 4
\definecolor{MSGoldAk4}{RGB}{255,192,0}
\definecolor{MSGoldAk4H80}{RGB}{255,242,204}
\definecolor{MSGoldAk4H60}{RGB}{255,229,153}
\definecolor{MSGoldAk4H40}{RGB}{255,217,102}
\definecolor{MSGoldAk4D25}{RGB}{191,143,0}
\definecolor{MSGoldAk4D50}{RGB}{128,96,0}

%Grau, Akzent 3
\definecolor{MSGrau3}{RGB}{165,165,165}
\definecolor{MSGrau3H80}{RGB}{237,237,237}
\definecolor{MSGrau3H60}{RGB}{219,219,219}
\definecolor{MSGrau3H40}{RGB}{201,201,201}
\definecolor{MSGrau3D25}{RGB}{123,123,123}
\definecolor{MSGrau3D50}{RGB}{82,82,82}

%Orange, Akzent 2
\definecolor{MSOrange2}{RGB}{237,125,49}
\definecolor{MSOrange2H80}{RGB}{251,228,213}
\definecolor{MSOrange2H60}{RGB}{247,202,172}
\definecolor{MSOrange2H40}{RGB}{244,176,131}
\definecolor{MSOrange2D25}{RGB}{196,89,17}
\definecolor{MSOrange2D50}{RGB}{131,60,11}

%Blau, Akzent 1
\definecolor{MSBlau1}{RGB}{68,114,196}
\definecolor{MSBlau1H80}{RGB}{217,226,243}
\definecolor{MSBlau1H60}{RGB}{180,198,231}
\definecolor{MSBlau1H40}{RGB}{142,170,219}
\definecolor{MSBlau1D25}{RGB}{47,84,150}
\definecolor{MSBlau1D50}{RGB}{31,56,100}

%Blaugrau, Text 2
\definecolor{MSBlauGrau2}{RGB}{68,84,106}
\definecolor{MSBlauGrau2H80}{RGB}{213,220,228}
\definecolor{MSBlauGrau2H60}{RGB}{172,185,202}
\definecolor{MSBlauGrau2H40}{RGB}{132,150,176}
\definecolor{MSBlauGrau2D25}{RGB}{50,62,79}
\definecolor{MSBlauGrau2D50}{RGB}{34,42,53}

%Hellgrau, Hintergrund 2
\definecolor{MSHellGrau2}{RGB}{231,230,230}
\definecolor{MSHellGrau2H80}{RGB}{208,206,206}
\definecolor{MSHellGrau2H60}{RGB}{174,170,170}
\definecolor{MSHellGrau2H40}{RGB}{118,113,113}
\definecolor{MSHellGrau2D25}{RGB}{59,56,56}
\definecolor{MSHellGrau2D50}{RGB}{23,23,23}

%Schwarz, Text 1
\definecolor{MSSchwarz1}{RGB}{0,0,0}
\definecolor{MSSchwarz1H80}{RGB}{127,127,127}
\definecolor{MSSchwarz1H60}{RGB}{89,89,89}
\definecolor{MSSchwarz1H40}{RGB}{64,64,64}
\definecolor{MSSchwarz1D25}{RGB}{38,38,38}
\definecolor{MSSchwarz1D50}{RGB}{13,13,13}

%Weiss, Hintergrund 1
\definecolor{MSWeiss1}{RGB}{255,255,255}
\definecolor{MSWeiss1H80}{RGB}{242,242,242}
\definecolor{MSWeiss1H60}{RGB}{217,217,217}
\definecolor{MSWeiss1H40}{RGB}{191,191,191}
\definecolor{MSWeiss1D25}{RGB}{166,166,166}
\definecolor{MSWeiss1D50}{RGB}{128,128,128}

%Einfarbige
\definecolor{MSLila}{RGB}{112,48,160}
\definecolor{MSDunkelblau}{RGB}{0,32,96}
\definecolor{MSBlau}{RGB}{0,112,192}
\definecolor{MSHellblau}{RGB}{0,176,240}
\definecolor{MSGrün}{RGB}{0,176,80}
\definecolor{MSHellgrün}{RGB}{146,208,80}
\definecolor{MSGelb}{RGB}{255,255,0}
\definecolor{MSOrange}{RGB}{255,192,0}
\definecolor{MSRot}{RGB}{255,0,0}
\definecolor{MSDunkelrot}{RGB}{192,0,0}

%----------------------------------------------------------------------------------------
%   SONSTIGE EINSTELLUNGEN
%----------------------------------------------------------------------------------------
\usepackage{float}
\usepackage{longtable}

%----------------------------------------------------------------------------------------
%   TABELLEN EINSTELLUNGEN
%----------------------------------------------------------------------------------------
\newcolumntype{R}[1]{>{\raggedleft\arraybackslash}p{#1}}        % Einstellung für Rechtsbündig
\renewcommand{\figurename}{Abbildung}
\renewcommand{\tablename}{Tabelle}
\renewcommand{\lstlistingname}{Code} 
\renewcommand{\lstlistlistingname}{}


%----------------------------------------------------------------------------------------
% DIPLOMARBEIT INFORMATION: DIESEN ABSCHNITT ÄNDERN!
%----------------------------------------------------------------------------------------

% Die folgenden Informationen werden in den folgenden Teilen verwendet:
% - Titelblatt
% - Impressum
% - Kurzfassung / Zusammenfassung
% - Meta-Informationen des PDF

\thesistitle{Projekt~Titel}                                 % Titel Diplomarbeit,                   Befehl: \ttitle
\thesistype{FACH - ATL}                                            % Art der Arbeit (z. B. Diplomarbeit)           \ttype
\thesisdate{2024-10-22}                                         % Datum der Abgabe                              \tdate
\thesislocation{Pratteln}                                   % Ort der Erstellung                            \tloc
\keywords{Beispiel1, Beispiel2, Beispiel3}             % Schlüsselwörter für die Arbeit                \keywordnames                          
\author{Max Mustermann}                                         % Authorenname (Ihr Name)                       \authorname
\authorclass{HE22}                                          % Klasse des Autors                             \authorcls
\degree{Dipl. Techniker/in HF Informatik}                   % Name des Studiengangs                         \degreename
\studyprogram{}                                             % Studiengang                                   \studyprog                                      
\studyprogramlink{}
                                                            % Link zum Studiengang                          \studyproglink
\supervisorA{Name Fachexperte}                                    % Name des Betreuers 1                          \supnameA
\supervisorAmail{}                                          % E-Mail Adresse des Betreuers 1                \supmailA
\supervisorAweb{}
                                                            %                                               \supwebA
\supervisorAUni{}                                           % Schule / Universistät des Betreuers 1         \supuniA
\supervisorAinfo{                                           % Formatierte Informationen über Betreuer 1     \supinfoA
    \supnameA\\
    \supuniA\\
    Email: \href{mailto:\supmailA}{\supmailA}\\
    Web: \href{\supwebA}{Link}
}

% Leer lassen, falls es keinen zweiten Betreuer gibt                                                        \supervisorB{}
\supervisorB{Name Experte}                                       % Name des Betreuers 2                          \supnameB
\supervisorBmail{}                           % E-Mail Adresse des Betreuers 2                \supmailB
\supervisorBweb{}          
                                                            %                                               \supwebB
\supervisorBUni{}                                           % Schule / Universistät des Betreuers 1         \supuniB
\supervisorBinfo{                                           % Formatierte Informationen über Betreuer 2     \supinfoB
    \supnameB\\
    \supuniB\\
    Email: \href{mailto:\supmailB}{\supmailB}\\
    Web: \href{\supwebB}{Link}
}

\university{Higher Technical School for Information and Communication Technology}          
                                                            % Schulname                                    \univname                                      
\universitygerman{Höhere Fachschule für Informations- und Kommunikationstechnologie}
                                                            % Schulname in Deutsch                         \univnameger
\department{}                                               % Abteilung, nicht notwendig                   \deptname                                  
\institute{}                                                % Institut, nicht notwendig                    \instname
\group{}                                                    % Forschungsgruppe, nicht notwendig            \groupname

% Links
\universitylink      {https://www.hf-ict.ch/}               %                                               \univlink
\universitylinkgerman{https://www.hf-ict.ch/}               %                                               \univlinkger
\departmentlink      {}                                     %                                               \deptlink
\institutelink       {}                                     %                                               \instlink
\grouplink           {}                                     %                                               \grplink



\AtBeginDocument{
\hypersetup{pdftitle=\ttitle}                               % Setzt den Titel der PDF-Datei auf Ihren Titel
\hypersetup{pdfauthor=\authorname}                          % Setzt den Autor der PDF-Datei auf den Angebeben (Ihren) Namen
\hypersetup{pdfkeywords=\keywordnames}                      % Setzt die Schlüsselwörter der PDF-Datei auf Ihre Schlüsselwörter
}

\begin{document}
\frontmatter                                                % Römische Seitennummerierung für die Seiten vor dem Inhalt
\pagestyle{plain}                                           % Standardmäßig wird der einfache Überschriftsstil verwendet, bis der Stil Arbeit 
                                                            % für den Hauptinhalt aufgerufen wird



\newenvironment{ItemZiele}
  {\begin{list}
        \setlength{\labelsep}{0em}
        \setlength{\itemsep}{2pt}
  }
{\end{list}}


%----------------------------------------------------------------------------------------
% TITELSEITE UND IMPRESSUM
%----------------------------------------------------------------------------------------
% !TEX root = ../main.tex

%----------------------------------------------------------------------------------------
% TITELSEITE
%----------------------------------------------------------------------------------------

\newgeometry{margin=1in}
\begin{titlepage}

% Die Titelseite möglichst unempfindlich machen gegeüber der Parskip Einstellungen
\setlength{\parskip}{0pt}

\begin{center}
\includegraphics[width=0.5\textwidth]{Bilder/hf-ict-Logo}

\vspace{0.5cm}
% Schulname
{\color{black}\LARGE \univnameger\par}
\vspace{0.3cm}

\vspace{1.5cm}

% Typ der Arbeit
{\LARGE \ttype\par}
\vspace{0.2cm}
\HRule 
\vspace{0.2cm}

% Titel der Arbeit
{\huge \bfseries \ttitle\par}
\vspace{0.2cm}
\HRule
\vspace{1.0cm}

% Autor
\begin{minipage}[t]{0.4\textwidth}
\begin{flushleft}
    \large
    \emph{Autor:} \\
    \authorname \\
    \vspace{0.5cm}
    \emph{Klasse:} \\
    \authcls
\end{flushleft}
\end{minipage}

% Fachexperte (A) und Experte (B)
\begin{minipage}[t]{0.4\textwidth}
\begin{flushright}
    \large
    \emph{Examinator/in:} \\
    \supnameA \\
    \vspace{0.5cm}
    \emph{Experte/in:} \\
    \supnameB \\
\end{flushright}
\end{minipage}

\vfill

{\large
\emph{Ort} \\
\tloc \\
\vspace{0.2cm}
\emph{Abgabedatum} \\
\tdate \\
\vspace{1.5cm}
}

% Infobox (für Disclaimer oder Sperrvermerk)
% Ausskommentieren, wenn dies benötigt wird.

%\vfill
%\begin{textbox}{red}{2}
%
%\end{textbox}

\vfill
\end{center}
\end{titlepage}
\restoregeometry

% Impressum der Arbeit, für hf-ict Arbeiten nicht notwendig
%\include{Vorderteil/impressum}


%----------------------------------------------------------------------------------------
% ABSTRACT
%----------------------------------------------------------------------------------------
% Römische Nummerierung für Abstract
\renewcommand{\thesection}{\Roman{section}}
\pagestyle{Diplomarbeit}
% !TEX root = ../main.tex

%----------------------------------------------------------------------------------------
% ABSTRACT (Management Summary)
%----------------------------------------------------------------------------------------
\section{Abstract}

Die Zusammenfassung entspricht einer Miniaturversion des gesamten Dokuments. Gliedere sie ähnlich: 
Beginne mit dem Kontext und der Motivation für das Projekt, einer kurzen Beschreibung der Methode und der verfügbaren Daten, 
Ihren Ergebnissen und den Schlussfolgerungen. Beschränke dich auf eine Seite!    

%----------------------------------------------------------------------------------------
% INHALTSVERZEICHNIS/ABBILDUNGEN/TABELLENSEITEN
%----------------------------------------------------------------------------------------
%  Auskommentieren, wenn eine der folgenden Seiten nicht benötigt wird:
% !TEX root = ../main.tex

%----------------------------------------------------------------------------------------
% INHALTSVERZEICHNIS
%----------------------------------------------------------------------------------------
\section{Inhaltsverzeichnis}                                    % Inhaltsverzeichnis mit römischer Nummerierung
\vspace{-2.1cm}
\renewcommand\contentsname{}                                    % Leerer Name für das Inhaltsverzeichnis
\begingroup
\let\clearpage\relax                                            % Verhindert den Seitenumbruch für das Inhaltsverzeichnis
{
  \hypersetup{linkcolor=black}
  \tableofcontents
}
\endgroup                         % Hauptinhaltsverzeichnis hinzufügen
%\listoffigures                                             % Liste der Abbildungen hinzufügen
%\listoftables                                              % Liste der Tabellen hinzufügen     
%\lstlistoflistings                                          % Liste der Codeblöcke      


%----------------------------------------------------------------------------------------
% EHRENWÖRTLICHE ERKLÄRUNG
%----------------------------------------------------------------------------------------
% !TEX root = ../main.tex

%----------------------------------------------------------------------------------------
% EHRENWÖRTLICHE ERKLÄRUNG
%----------------------------------------------------------------------------------------

\section{Ehrenwörtliche Erkl\"arung}

Ich erkläre und bestätige mit meiner Unterschrift, dass ich die vorliegende, im Rahmen 
des Qualifikationsverfahrens der hf-ict geforderte ATL, eigenständig verfasst
habe.

Die wörtlich oder inhaltlich den im Literaturverzeichnis aufgeführten Quellen und 
Hilfsmitteln entnommenen Stellen sind in der Arbeit als Zitat bzw. Paraphrase 
kenntlich gemacht.

\vspace{1cm}

\noindent Name:\\
\rule[0.5em]{30em}{0.5pt}                   % Linie für die Unterschrift
 
\noindent Ort, Datum:\\
\rule[0.5em]{30em}{0.5pt}                   % Linie für das Datum


%----------------------------------------------------------------------------------------
% VORWORT
%----------------------------------------------------------------------------------------
%\include{Vorderteil/vorwort}


%----------------------------------------------------------------------------------------
% GLOSSAR
%----------------------------------------------------------------------------------------
% !TEX root = ../main.tex

%----------------------------------------------------------------------------------------
% GLOSSAR
%----------------------------------------------------------------------------------------
\section{Glossar}

Hier kann der Glossar, nach eigenem Gusto, erstellt werden.
Es wird diesbezüglich kein Beispiel mitgeliefert.




%----------------------------------------------------------------------------------------
%  INHALT DER DIPLOMARBEIT - KAPITEL
%----------------------------------------------------------------------------------------
\mainmatter
\renewcommand{\thesection}{\arabic{section}{}}              % Beginn der numerischen (1,2,3...) Seitennummerierung
\setcounter{section}{0}                                     % Zurücksetzen des Counters für die Zählung
\pagestyle{Diplomarbeit}                                    % Rückführung der Seitenköpfe auf den "Diplomarbeit"-Stil

\include{Hauptteil/Kapitel1}
\include{Hauptteil/Kapitel2}
\include{Hauptteil/Kapitel3}
\include{Hauptteil/Kapitel4}
\include{Hauptteil/Kapitel5}

% Fügen Sie die Kapitel der Dissertation als separate Dateien in den Ordner Kapitel ein.
% Heben Sie die Kommentare in den Zeilen auf, während Sie die Kapitel schreiben


%----------------------------------------------------------------------------------------
% INHALT DER DIPLOMARBEIT - ANHÄNGE
%----------------------------------------------------------------------------------------
\pagestyle{Diplomarbeit}
%\appendix                                                  % Hinweis an LaTeX, dass es sich bei den folgenden "Kapiteln" um Anhänge handelt
\renewcommand{\thesection}{\Alph{section}}                  % Alpha Nummerierung für Anhänge
\setcounter{section}{0}                                     % Zurücksetzen des Counters für die Zählung

% !TEX root = ../main.tex

%----------------------------------------------------------------------------------------
% ANHANG A - QUELLENVERZEICHNIS
%----------------------------------------------------------------------------------------
\section*{Anhänge}
\section{Quellenverzeichnis}
% Definiere normale Schriftart auch für Quellenverzeichnis
\urlstyle{same}
%Definiere Umbrüche
\setcounter{biburllcpenalty}{7000}
\setcounter{biburlucpenalty}{8000}
\begingroup
\let\clearpage\relax                                            % Verhindert den Seitenumbruch für das Quellenverzeichnis
\printbibliography[heading=none]
\endgroup
% Zeigt auf die Hauptdatei. Muss am Anfang der Datei stehen.
% !TEX root = ../main.tex

%----------------------------------------------------------------------------------------
% ANHANG B - Abbildungsverzeichnis
%----------------------------------------------------------------------------------------
\renewcommand\listfigurename{}                          % Namensänderung für das Abbildungsverzeichnis
\section{Abbildungsverzeichnis} 
\vspace{-2.1cm}
\begingroup
\let\clearpage\relax                                    % Verhindert den Seitenumbruch für das Tabellenverzeichnis
{
    \hypersetup{linkcolor=black}
    \listoffigures                                      % Liste der Abbildungen hinzufügen
} 
\endgroup
% !TEX root = ../main.tex

%----------------------------------------------------------------------------------------
% ANHANG C - Tabellenverzeichnis
%----------------------------------------------------------------------------------------
\renewcommand\listtablename{}                           % Namensänderung für das Abbildungsverzeichnis
\section{Tabellenverzeichnis} 
\vspace{-2.1cm}
\begingroup
\let\clearpage\relax                                    % Verhindert den Seitenumbruch für das Tabellenverzeichnis
{
    \hypersetup{linkcolor=black}
    \listoftables                                       % Liste der Tabellen hinzufügen  
}
\endgroup
% !TEX root = ../main.tex

%----------------------------------------------------------------------------------------
% ANHANG D - Codeblockverzeichnis
%----------------------------------------------------------------------------------------
\renewcommand\listtablename{}                           % Namensänderung für das Codeblockverzeichnis
\section{Codeblockverzeichnis} 
\vspace{-2.1cm}
\begingroup
\let\clearpage\relax                                    % Verhindert den Seitenumbruch für das Codeblockverzeichnis
{
    \hypersetup{linkcolor=black}
    \lstlistoflistings                                       % Liste der Codeblöcke hinzufügen  
}
\endgroup

%----------------------------------------------------------------------------------------
% LITERATURVERZEICHNIS
%----------------------------------------------------------------------------------------


%----------------------------------------------------------------------------------------

\end{document}  
