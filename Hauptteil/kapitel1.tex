% Zeigt auf die Hauptdatei. Muss am Anfang der Datei stehen.
% !TEX root = ../main.tex

%----------------------------------------------------------------------------------------
% KAPITEL 1
%----------------------------------------------------------------------------------------

\section{Struktur}
\label{sec:Struktur}

In der ATL und für die Diplomarbeit wird die Kapitelführung mit \textit{section} und textit{subsection} gestaltet.
Dies kann jedoch auch selbstständig geändert werden.
Pro Kapitel wird jeweils eine eigene Datei erstellt. Dies führt jedoch auch dazu, dass jedes neue Kapitel auf einer eigenen Seite beginnt.

Dieses Dokument dient nicht dazu, \LaTeX{} zu erläutern, sondern stellt gewisse Punkte zur Verfügung, welche häufig in der Arbeit benutzt werden können.

\section{Sonderzeichen}
\label{sec:Sonderzeichen}

Einige spezielle Zeichen, benötigen spezielle Befehle.
Dazu gehören auch die Klammern. Vor allem, wenn diese in einer speziellen Schriftart ausgeführt werden sollen.

Dazu gehören auch unter anderem die Anführungszeichen. Diese können mit \verb|``| begonnen und mit beendet \verb|''| werden.

Alternative Anführungszeichen sind: \frqq \verb|\frqq|, \flqq \verb|\flqq|, \frq \verb|\frq| sowie \flq \verb|\flq|.

Weitere Sonderzeichen sind Klammern.

\begin{enumerate}[parsep=0pt]
  \item Linke Klammer (\, \LaTeX{} Befehl: \verb| (\, |
  \item Rechte Klammer )\, \LaTeX{} Befehl: \verb| (\, |
  \item Geschweifte linke Klammer \{ \LaTeX{} Befehl: \verb| \{ |
  \item Geschweifte rechte Klammer \} \LaTeX{} Befehl: \verb| \} |
  \item Eckige linke Klammer [\, \LaTeX{} Befehl: \verb| [\, |
  \item Eckige rechte Klammer ]\, \LaTeX{} Befehl: \verb| ]\, |
\end{enumerate}

\section{Aufzählungen}
\label{sec:Aufzählungen}

Aufzählungen sind sinnvoll, wenn sie mit möglichst wenig Abstand durchgeführt werden. So sehen sie ``natürlich'' aus.

\begin{itemize}[parsep=0pt]
    \item Beispiel 1
    \item Beispiel 2
    \item Beispiel 3
\end{itemize}

\section{Quellen}
\label{sec:Quellen}

Ebenfalls können Quellen \parencite{beispiel_atl} eingebunden werden.

\section{Tabellen}
\label{sec:Tabellen}

Tabellen können auf unterschiedliche Weisen eingebunden werden. Bei grossen Tabellen empfiehlt sich \textit{longtable} zu verwenden.

\begin{longtable}{p{7.0cm}p{5.0cm}}
    \hline
    \rowcolor{MSBlauAk5H60}
    \textbf{Überschrift 1} & \textbf{Überschrift 2} \\ \hline
    Daten 1 & Daten 2 \\ \hline
    Daten 1 & Daten 2 \\ \hline
    Daten 1 & Daten 2 \\ \hline
    \caption{Beispiel Tabelle}\\  
\end{longtable}
%----------------------------------------------------------------------------------------