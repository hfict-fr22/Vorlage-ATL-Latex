% Zeigt auf die Hauptdatei. Muss am Anfang der Datei stehen.
% !TEX root = ../main.tex

%----------------------------------------------------------------------------------------
% KAPITEL 2
%----------------------------------------------------------------------------------------

\section{Code Abschnitte}
\label{sec:Code-Abschnitte} 

Das Paket \href{https://www.overleaf.com/learn/latex/Code\_listing}{\code{listings}} erlaubt es, einfach Code Abschnitte zu benutzen. Dies erfolgt mit dem Befehl \verb|\lstinputlisting[language=name]{path/to/file}|. Die verfügbaren unterstützen Sprachen können \href{https://www.overleaf.com/learn/latex/Code\_listing#Supported\_languages}{hier} nachgeschlagen werden.

\lstinputlisting[language=Python,caption=Externe Datei: Code/beispiel.py]{Code/beispiel.py}

Es ist ebenso möglich, den Code direkt in \LaTeX einzubinden:

\begin{lstlisting}[language=C++]
#include <stdio>
void hello_world(void){
   std::cout << "Hello World!" << std::endl;
}    
\end{lstlisting}

\section{Bilder}
\label{sec:Bilder}

Bilder können ebenfalls eingebunden werden.

\begin{figure}[th]
\centering
\includegraphics{Bilder/Electron}
\decoRule
\caption[Ein Elektron]{Ein Elektron (\,künstlerischer Eindruck)\,.}
\label{fig:Electron}
\end{figure}


%----------------------------------------------------------------------------------------