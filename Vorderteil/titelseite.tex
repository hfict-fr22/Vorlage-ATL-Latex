% !TEX root = ../main.tex

%----------------------------------------------------------------------------------------
% TITELSEITE
%----------------------------------------------------------------------------------------

\newgeometry{margin=1in}
\begin{titlepage}

% Die Titelseite möglichst unempfindlich machen gegeüber der Parskip Einstellungen
\setlength{\parskip}{0pt}

\begin{center}
\includegraphics[width=0.5\textwidth]{Bilder/hf-ict-Logo}

\vspace{0.5cm}
{\color{black}\LARGE \univnameger\par}                % Schule
\vspace{0.3cm}

\vspace{1.5cm}                            
{\LARGE \ttype\par}                                             % Typ der Arbeit
\vspace{0.2cm}
\HRule 
\vspace{0.2cm}
{\huge \bfseries \ttitle\par}                                   % Titel der Arbeit
\vspace{0.2cm}  
\HRule
\vspace{1.0cm}

% Autor 
\begin{minipage}[t]{0.4\textwidth}
\begin{flushleft} 
    \large
    \emph{Autor:}\\
    \authorname\\
    \vspace{0.5cm}
    \emph{Klasse:}\\
    \authcls
\end{flushleft}
\end{minipage}
% Examinator (A) und Experte (B)
\begin{minipage}[t]{0.4\textwidth}
\begin{flushright} 
    \large
    \emph{Examinator/in:} \\
    \supnameA \\
    \vspace{0.5cm}
    \emph{Experte/in:} \\
    \supnameB \\
\end{flushright}
\end{minipage}

\vfill

{\large
\emph{Ort}\\
\tloc\\
\vspace{0.2cm}
\emph{Abgabedatum}\\
\tdate\\
% Abstand ist initial 1.5cm, bei Banner z. B. 0.5cm
\vspace{0.5cm}
}

\vfill
\begin{textbox}{red}{2}
    Löschen Sie diese Infobox, wenn Sie nicht benötigt wird. Hier kann ein Disclaimer oder Sperrvermerk eingefügt werden.
\end{textbox}

\vfill
\end{center}
\end{titlepage}
\restoregeometry